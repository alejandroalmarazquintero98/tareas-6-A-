\documentclass[12pt,a4paper]{article}
\usepackage[utf8]{inputenc}
\usepackage{amsmath}
\usepackage{amsfonts}
\usepackage{amssymb}
\usepackage{makeidx}
\usepackage{graphicx}
\usepackage[left=2cm,right=2cm,top=2cm,bottom=2cm]{geometry}

\author{ALMARAZ QUINTERO ALEJANDRO}
\title{Identificar las aplicaciones de los manipuladores paralelos}



\begin{document}
\maketitle

\includegraphics[scale=0.25]{../../../Downloads/CUeq3LKx.png} 
\clearpage

\section{Las aplicaciones de los manipuladores paralelos}
La presencia de robots manipuladores es algo frecuente en prácticamente
cualquier sector industrial. La necesidad de aumentar la producción, disminuir
los costes de un proceso determinado, etc. hacen que muchas veces se prefiera
utilizar este tipo de mecanismos para sustituir a personal que, aunque quizás
estuviese capacitado para la realización de unas tareas determinadas.
Estas razones puramente económicas no serían, sin embargo, las únicas por
las que la utilización de este tipo de dispositivos estaría justificada. Los robots
manipuladores también son diseñados con objeto de realizar trabajos que, bien
por peligrosos, monótonos, necesitados de una alta precisión y repetitibilidad,
o bien por estar destinados a soportar cargas elevadas, liberen al hombre de
realizar estos cometidos.

La utilización de los diferentes sistemas mecánicos ha ido variando con el
paso del tiempo, hasta llegar a la situación en la que hoy en día nos encontramos. Habitualmente, el empleo de mecanismos es y ha sido necesario en la
transmisión y conversión de movimientos de un tipo a otro (rotación-rotación,
rotación-traslación, etc.) o la obtención de un tipo de terminado de movimientos y trayectorias de mayor o menor complejidad.


\includegraphics[scale=0.6]{../../../Pictures/Aplicaciones.jpg} 

Los manipuladores paralelos de baja movilidad son, sin duda, un tipo de robots quepresentan importantes ventajas, en ciertos aspectos, sobre los denominadosmanipuladores serie, especialmente cuando se requiere mayor rigidez, aún a costa deuna reducción del espacio de trabajo del manipulador paralelo.El desarrollo de nuevas arquitecturas de este tipo de manipuladores ha recibido notableimpulso desde la segunda mitad del siglo XX, si bien su implantación industrial tienenotables resistencias debido a la complejidad de la cinemática y a la dificultad decontrol en tiempo real y limitaciones de espacio de trabajo.Esta Tesis Doctoral se centra en el desarrollo de una sistemática de generación dediferentes arquitecturas de manipuladores paralelos de baja movilidad, es decir, de menos de seis grados de libertad.

Existe una gran variedad de funciones que pueden realizar los robots paralelos. Hasta el momento se han citado algunas como la manipulación de componentes (pick & place), los simuladores de vuelo, la orientación de plataformas (desde aplicaciones espaciales como antenas hasta médicas como microscopios de precisión), y la máquina-herramienta (mecanizado de piezas como el fresado, torneado, escariado o taladrado). Otras aplicaciones son las operaciones quirúrgicas de precisión, ensamblado de componentes electrónicos o los micromanipuladores, capaces de realizar movimientos de unos pocos nanómetros.

En cuanto a los campos de aplicación con más posibilidades cabe citar el aeronáutico y la industria automovilística. El primero de ellos tiene dos vertientes, referida la primera de ellas al mecanizado de piezas. La segunda vertiente se
refiere al tratamiento de piezas de gran envergadura, cuasi-acanaladas con poca curvatura en sentido longitudinal, las cuales requieren plataformas de 4 ó 5
GDL. En cuanto a la industria automovilística y auxiliar, las mayores necesidades residen en el mecanizado de componentes en las transmisiones de potencia como por ejemplo cajas de cambio


\end{document}